\documentclass[12pt,a4paper]{article}
\usepackage[utf8]{inputenc}
\usepackage{fullpage}
\usepackage{inconsolata}
\usepackage[protrusion]{microtype}
\usepackage[table]{xcolor}

% Typesetting code
\definecolor{base-blue}{RGB}{13, 91, 201}
\definecolor{keywords}{RGB}{129, 19, 8}    % red
\definecolor{identifiers}{RGB}{19, 8, 129} % blue
\definecolor{strings}{RGB}{8, 129, 19}     % green
\definecolor{comments}{RGB}{104, 104, 110} % gray
\definecolor{error}{RGB}{220, 50, 47}      % bright red
\usepackage{listings}	% listlisting environment
\usepackage{upquote}
\lstset{
  basicstyle=\ttfamily\small\bfseries,
  breakatwhitespace=false,
  breaklines=true,
  captionpos=b,
  commentstyle=\ttfamily\color{comments}\bfseries,
  escapeinside={\$}{)},
  identifierstyle=\color{identifiers},
  keywordstyle=\ttfamily\color{keywords},
  language=Python,
  mathescape=true,
  numbers=left,
  stepnumber=1,
  showspaces=false,
  showstringspaces=false,
  showtabs=false,
  stringstyle=\ttfamily\color{strings},
  tabsize=2,
  aboveskip=-0.3em,
  belowskip=-0.3em,
  xleftmargin=2.15em,
}
% Frames around code lstlistings
\usepackage[framemethod=TikZ]{mdframed}
\usepackage{etoolbox}
\mdfdefinestyle{listingstyle}{%
	innerleftmargin=0pt,
	innerrightmargin=0pt,
	rightmargin=0pt,
	linewidth=0pt,
	backgroundcolor=base-blue!5!white,
}
\BeforeBeginEnvironment{lstlisting}{\begin{mdframed}[style=listingstyle]}
\AfterEndEnvironment{lstlisting}{\end{mdframed}}

\begin{document}

\section*{The good (correct and proper style)}

\begin{lstlisting}
def is_prime(number):
    """Primality test by trial division."""
    if number < 2:
        return False
    if number == 2:
        return True
    if number % 2 == 0:
        return False
    for candidate in range(3, int(sqrt(number)) + 1):
        if number % candidate == 0:
            return False
    return True
\end{lstlisting}

\begin{lstlisting}
def reverse(string):
    """Revert the order of characters in <string>."""
    result = ""
    for character in string:
        result = character + result
    return result
\end{lstlisting}

\begin{lstlisting}
class Dragon:
    """Represents a dragon in a fantasy computer game."""
    def __init__(self, param_name, param_level, param_life):
        self.name = param_name
        self.level = param_level
        self.life = param_life

object_smaug = Dragon("Smaug", 84, 112233)
object_norbert = Dragon("Norbert", 11, 2048)
\end{lstlisting}

\section*{The bad (syntactic, semantic, or logical error)}

\begin{lstlisting}
def sum_while(how_many):
    """Sum the first <how_many> natural numbers."""
    result = 0
    while how_many > 0:
        result = result + how_many
    how_many = how_many - 1
    return result
\end{lstlisting}
% Line 6 should be indented within the while cycle

\begin{lstlisting}
def recursive_list_sum(numbers):
    return numbers[0] + list_sum(numbers[1:])
\end{lstlisting}
% Missing halting case for an empty list

\begin{lstlisting}
def find_maximum(lst):
    maximum = lst[0]
    for number in lst:
        if number < maximum:
            maximum = number
    return maximum
\end{lstlisting}
% Line 4 should be: if number > maximum

\begin{lstlisting}
class Player:
    """Represents the player's character."""
    def __init__(self, name, health_max):
        self.name = name
        self.health = health_max
        self.health_max = health_max
        self.level = 1


def level_up():
    """Level up the player."""
    self.level += 1
    self.health_max += 1
    self.health += health_max
    print("{} feels stronger!".format(self.name))
\end{lstlisting}
% Missing self parameter definition on line 10

\section*{The ugly (correct but improper style)}

\begin{lstlisting}
def can_buy_beer(persons_age):
    if persons_age >= 18:
        return True
    else:
        return False
\end{lstlisting}
% Use a one-liner: return persons_age >= 18

\begin{lstlisting}
def find_maximum(lst):
    maximum = lst[0]
    for index in range (1, len(lst)):
        if lst[index] > maximum:
            maximum = lst[index]
    return maximum
\end{lstlisting}
% Python style suggests iterating for each element instead of using range, also a space after range on line 3

\begin{lstlisting}
def letter_distribution_analysis(string):
    dictionary = {}
    for letter in string:
        if letter not in dictionary: dictionary[ letter ] = 1
        else: dictionary[ letter ] += 1
    print_dictionary(dictionary)
\end{lstlisting}
% Multiple statements on one line are generally discouraged, spaces inside brackets

\end{document}