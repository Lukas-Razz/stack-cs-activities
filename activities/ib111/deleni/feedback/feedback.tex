\documentclass[12pt,a5paper]{article}
\usepackage[czech]{babel}
\usepackage[utf8]{inputenc}
\usepackage[T1]{fontenc}
\usepackage{geometry}
\usepackage{lmodern}
\usepackage{multicol}
\usepackage{listings}
\textwidth 128mm \textheight 210mm
\topmargin -3cm
\oddsidemargin -1.5cm
\begin{document}
\pagestyle{empty}

\section*{Zpětná vazba (na cvičení)}
Tady a teď máte jedinečnou příležitost změnit předmět IB111 a udělat jej pro
budoucí generace lepší. Využijte této příležitosti a zamyslete se nad tím, co
se vám v~předmětu líbilo, nelíbilo, co zachovat a co změnit. Více návodných
otázek hledejte na zadní straně.

\newpage

\section*{Zpětná vazba (na cvičení)}
Tady a teď máte jedinečnou příležitost změnit předmět IB111 a udělat jej pro
budoucí generace lepší. Využijte této příležitosti a zamyslete se nad tím, co
se vám v~předmětu líbilo, nelíbilo, co zachovat a co změnit. Více návodných
otázek hledejte na zadní straně.

\newpage

\section*{Otázky}

Následující otázky můžete chápat jako inspiraci pro psaní zpětné vazby na
druhou stranu. Rozhodně není třeba je jich striktně držet.

\begin{itemize}
	\item Přišla ti cvičení smysluplná? Jak by vypadalo ideální
		  cvičení (např. míra opakování tématu, programování, demonstrativních
		  ukázek cvičícím na projektoru, ...)?
	\item Které cvičení se ti líbilo nejvíc a které nejméně (a proč)?
	\item Co si myslíš o párovém programování? Jaké to pro tebe mělo
		  výhody/nevýhody oproti samostatnému programování? Byl bys radši,
		  kdyby se na cvičeních programovalo samostatně?
	\item Která aktivita pro rozřazení do dvojic se ti líbila nejvíce/nejméně?
		  Nebylo by lepší, kdybych vás rozřazoval náhodně bez aktivit, nebo vás
		  nechal rozdělit do libovolných dvojic? Máš nějaký nápad/námět na
		  aktivitu pro rozřazení do dvojic?
	\item Měl jsi vždy všechny potřebné informace, komunikoval s tebou cvičící?
	\item Vyhovoval ti web cvičení jako platforma pro předávání informací?
	\item Domácí úlohy. Styl? Náročnost? Pracnost?
	\item Vnitra. Styl? Náročnost? Pracnost?
	\item Chtěl bys více akčních aktivit podobných Hanojským věžím?
	\item Která část předmětu ti přišla nejtěžší (nebo třeba nejhůře
	      vysvětlená, příliš málo úloh ve sbírce, ...)?
	\item Co je ta nejlepší věc, kterou sis ze cvičení odnesl?
\end{itemize}

\newpage

\section*{Otázky}

Následující otázky můžete chápat jako inspiraci pro psaní zpětné vazby na
druhou stranu. Rozhodně není třeba je jich striktně držet.

\begin{itemize}
	\item Přišla ti cvičení smysluplná? Jak by vypadalo ideální
		  cvičení (např. míra opakování tématu, programování, demonstrativních
		  ukázek cvičícím na projektoru, ...)?
	\item Které cvičení se ti líbilo nejvíc a které nejméně (a proč)?
	\item Co si myslíš o párovém programování? Jaké to pro tebe mělo
		  výhody/nevýhody oproti samostatnému programování? Byl bys radši,
		  kdyby se na cvičeních programovalo samostatně?
	\item Která aktivita pro rozřazení do dvojic se ti líbila nejvíce/nejméně?
		  Nebylo by lepší, kdybych vás rozřazoval náhodně bez aktivit, nebo vás
		  nechal rozdělit do libovolných dvojic? Máš nějaký nápad/námět na
		  aktivitu pro rozřazení do dvojic?
	\item Měl jsi vždy všechny potřebné informace, komunikoval s tebou cvičící?
	\item Vyhovoval ti web cvičení jako platforma pro předávání informací?
	\item Domácí úlohy. Styl? Náročnost? Pracnost?
	\item Vnitra. Styl? Náročnost? Pracnost?
	\item Chtěl bys více akčních aktivit podobných Hanojským věžím?
	\item Která část předmětu ti přišla nejtěžší (nebo třeba nejhůře
	      vysvětlená, příliš málo úloh ve sbírce, ...)?
	\item Co je ta nejlepší věc, kterou sis ze cvičení odnesl?
\end{itemize}

\end{document}
