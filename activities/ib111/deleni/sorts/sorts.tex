\documentclass[12pt,a4paper]{article}
\usepackage[czech]{babel}
\usepackage[utf8]{inputenc}
\usepackage[T1]{fontenc}
\usepackage{lmodern}
\usepackage{multicol}
\textwidth 19cm \textheight 24.6cm
\topmargin -1.3cm
\oddsidemargin -1.5cm
\begin{document}
\pagestyle{empty}

\setlength\parindent{0pt}
\setlength{\columnsep}{60pt}
\begin{multicols}{2}

\Large

Bubble sort\\

Řadící algoritmus založený na prohazování sousedních prvků dokud není celý
seznam seřazený.\\

Select sort\\

Dělí seznam na seřazenou a neseřazenou část. Z~neseřazené části vždy
vybere minimum a to prohodí s~prvním prvkem neseřazené části.\\

Insert sort\\

Dělí seznam na seřazenou a neseřazenou část. Prvky z~neseřazené části postupně
zařazuje na správné místo do seřazené části.\\

Quick sort\\

Rychlý řadicí algoritmus s~logaritmickou složitostí, ve kterém důležitou roli
hraje správná volba pivota.\\

Merge sort\\

Rekurzivní řadící algoritmus založený na seřazení 2 částí seznamu, které se
následně spojují dohromady.\\

Counting sort\\

Lineární řadící algoritmus, který lze použít na data z~omezeného rozsahu.\\

Radix sort\\

Řadící algoritmus založený na rozkladu čísel na cifry.\\

Bogo sort\\

Řadící algoritmus založený na náhodném prohazování prvků a opakované kontrole
seřazenosti seznamu.\\

\rule{\linewidth}{1pt}\\

Řadící algoritmus s~logaritmickou časovou složitostí.

\end{multicols}

\end{document}
